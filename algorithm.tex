
\section{PS algorithm}
\secL{algorithm}
In this section we present PS, the algorithm that we propose to perform the Design Space Exploration (DSE). As any other DSE algorithm, its input is the parameterized system $S$ and a benchmark application $b$, while the output is an approximation of the Pareto-optimal set $\mathcal{P}\left(S,b\right)$.   
PS is iterative and we will describe, separately, the initialization phase, the operations performed inside each iteration and the condition that triggers the termination.

At a glance, at each iteration simulations are performed. Regions are classified based on the innovation introduced by their configurations in the Pareto-front so far calculated. The most interesting regions are split, so that each sub-region will receive evaluation effort in the next iteration. On the contrary, uninteresting regions are merged.

By a slightly abuse of notation, considering a subset $I$ of the configuration space, we will use  $\mathcal{P}(I,b)$ to denote the non-dominated feasible configurations in $I$, i.e.
	\begin{align}
		\mathcal{P}(I,b)=
		\left\{ \mathbf{c} \in I \cap \mathcal{C}^*(S) | \nexists \ \mathbf{c}' \in I \cap \mathcal{C}^*(S), \mathbf{c}' \succ \mathbf{c} \right\}
	\end{align}
Since in each run of the algorithm we will consider only one benchmark application $b\in\mathcal{B}$, we will omit it, replacing, for example, $\mathcal{P}(I,b)$ with  $\mathcal{P}(I)$.

Each iteration is called \emph{era}. The $i$-th era is characterized by: 
	\begin{itemize}
	\item the set of regions $\mathcal{R}_{i}$ in which the configuration space
	$mathcal{C}(S)=V_{1}\times\dots\times V_{n}$ is partitioned
	\item an approximation $\mathscr{P}_{i}$ of the Pareto-optimal set.
	\item a set $test_{i}$ of at most $K$ configurations evaluated
	in the era.
	\end{itemize}
$K$ is a parameter that must be set before the algorithm runs.
Note that $\mathcal{R}_{i}$ is a partition of the configuration space, i.e.
	\begin{align}\begin{array}{l}
		\bigcup_{R\in\mathcal{R}_i} R = \mathcal{C}(S) \mbox{ and}\\
		R' \cap R'' = \emptyset, \forall R',R'' \in \mathcal{R}_i
	\end{array}\end{align}

We now describe formally PS.

\subsection{Initialization}
We fix $\mathcal{R}_0$, $\mathscr{P}_0$ and $test_0$ for the era $0$ as follows. Era $0$ has only one region that is the whole parameter space,
i.e. $\mathcal{R}_{0}=\left\{ V_{1}\times\dots\times V_{n}\right\} $.
A random set $test_{0}$ of $K$ configurations is evaluated. The
Pareto set $\mathscr{P}_{0}=\mathscr{P}\left(test_{0}\right)$ is
calculated.



\subsection{Iteration steps}

For $era_{i}$ with $i>0$ the following operations must be performed.
\begin{enumerate}
\item \label{pers02.enu:K}
In each era $i$, a set $test_i$ of $K$ configurations will be evaluated. We call them \emph{test configurations} and selecting them is exactly the goal of this step.
We select only \emph{new} configurations, i.e. the ones that have not been evaluated in past eras. In other words, $test_i$ must not contain $\bigcup_{j=0}^{i-1}test_{j}$.
Since all regions $R$ in the configuration space partition $\mathcal{R}_{i}$ must be properly explored, we have to distribute the test configurations among all these regions. For this reason, $\frac{K}{\left|\mathcal{R}_{i}\right|}$ configurations are randomly extracted from the new configurations of each region $R\in\mathcal{R}_{i}$. We use $test_{R,i}$ to denote the set of the selected configurations and, obviously, $\bigcup_{R\in\mathcal{R}_{i}} test_{R,i} = test_i$.
 In case the new configurations in $R$ are less then $\frac{K}{\left|\mathcal{R}_{i}\right|}$, all of them are selected. In this case $test_{R,i}=R\setminus\bigcup_{j=0}^{i-1}test_{j}$, where  $\bigcup_{j=0}^{i-1}test_{j}$ is the set of the configurations already in the past.\footnote{Note that if such regions exist, the total number of evaluations in era $i$ will be less than $K$}
\item All configurations on $test_{i}$ are simulated.
\item The Pareto set $\mathscr{P}_{i}$ for $era_{i}$ is defined as the set of non-dominated configurations among the ones evaluated so far, i.e.
	\[
	\mathscr{P}_{i} \triangleq
	\mathscr{P}\left(\bigcup_{j\leq i}test_j\right)=
	\mathscr{P}\left(test_{i}\cup\mathscr{P}_{i-1}\right)
	\]
where the last equality can easily be proved.

\item \label{pers02.enu:novelty_score_of_a_configuration}
The test configurations that are dominated, i.e. do not belong to $\mathscr{P}_i$, are neglected, while an \emph{innovation score} is given to the other ones, computed on the base of their separation from the non-dominated configurations of the past eras. Recalling that $s$ is the separation (see \defR{separation}), the innovation score of a non-dominated test configuration $\mathbf{c}^*$ is defined as
	\[
	is\left(\mathbf{c}^*\right)=\min\left\{ \left.s\left(\mathbf{c}\rightarrow\mathbf{c} \right)\right|\mathbf{c}\in\mathscr{P}_{i-1}\right\} 
	\]
In other words, a configuration $\mathbf{c}^*$ is characterized by as much innovation
as more separated it is from the configurations of the previous era.

\item Every region $R\in\mathcal{R}_{i}$ is given an innovation score defined
as the sum of the innovation scores of its non-dominated test configurations:
	\[
	is\left(R\right)=\sum\left\{ \left.is\left(\mathbf{c}\right)\right|\mathbf{c}\in\mathscr{P}_{i}\cap R\right\} 
	\]

\item The total innovation and average score for the entire era is calculated:
	\begin{align} \begin{array}{l}
	is_{TOT,i}=\sum_{R\in\mathcal{R}_{i}}is\left(R\right) \\
	is_{av,i}=is_{TOT,i} / |\mathcal{R}_i|
	\end{array} \end{align}
	

\item The configuration space partition $\mathcal{R}_{i}$ is divided in three subsets, that we will comment later:

	\begin{enumerate}
	\item \emph{high innovation regions}, whose innovation score exceeds the average score of the previous era by a factor of $\alpha$, i.e.
	\[
	\mathcal{R}_{i,h}=\left\{ \left.R\in\mathcal{R}_{i}\right|is\left(R\right)>\alpha\cdot is_{av,i-1}\right\} 
	\]
	 where $\alpha$ is a constant defined by the designer (for example
	$\alpha=1.2$). 
	\item \emph{low innovation regions}, whose score is positive but modest, i.e.
	\[
	\mathcal{R}_{i,l}=\left\{ \left.R\in\mathcal{R}_{i}\right|0<is\left(R\right)\le\alpha\cdot is_{av,i-1}\right\} 
	\]
	\item \emph{no innovation regions}, whose innovation is null, i.e. 
	\[
	\mathcal{R}_{i,n}=\left\{ \left.R\in\mathcal{R}_{i}\right|is\left(R\right)=0\right\} 
	\]
	\end{enumerate}

\item The goal of this step is preparing the regions that will compose the configuration space partition of the next era, i.e. $\mathscr{R}_{i+1}$. The following operations are performed:
	\begin{enumerate}
		\item Each high innovation region $R\in\mathscr{R}_{i,h}$ is split to obtain $\psi(R)$, as stated in \defR{splitting-a-region}. We use $\psi\left(\mathscr{R}_{i,h}\right)$ to denote the set of regions obtained as explained.
		\item Low innovation regions are left unchanged.
		\item No innovation regions are merged, two by two. To do so, pairs of no innovation contiguous regions $(R_1,R_2)$ are selected. We use $\mathscr{M}_i$ to represent the set of these pairs. This set is constructed such that pairs are independent, i.e. there cannot be two of them sharing a region. In other words, taking whatever $(R_1,R_2),(R_3,R_4) \in \mathscr{M}_i$, $R_i \neq R_j, \forall i,j\in\lbrace1,2,3,4\rbrace, i\neq j$. 
We merge each pair of these regions, i.e. $\forall (R_1,R_2)\in\mathscr{M}_i$ we calculate $\mu(R_1,R_2)$, according to \defR{merging-regions}.
We stress that the regions that belong to each selected pair must be contiguous, otherwise it is impossible to merge them. We use $\mu(\mathscr{M}_i)$ to denote the set of the merged regions, i.e. $\mu(\mathscr{M}_i) = \bigcup_{(R_1,R_2)\in\mathscr{M}_i} \mu(R_1,R_2)$
	\end{enumerate}


\item The set of regions $\mathcal{R}_{i+1}$ for the following era is composed of the sub-regions resulting from splitting operations ($\psi\left(\mathscr{R}_{i,h}\right)$), the low innovation regions unchanged $\mathcal{R}_{i,l}$, the result of the merging operations ($\mu(\mathscr{M}_i))$ and the no innovation regions that was not possible to merge $\left(\mathcal{R}_{i,n}\setminus\bigcup_{\left(R_{1},R_{2}\right)\in\mathcal{C}}\left\{ R_{1},R_{2}\right\} \right)$. Formally:
	\begin{align} 
	\mathcal{R}_{i+1} \triangleq
	\psi\left(\mathscr{R}_{i,h}\right) \cup \mathcal{R}_{i,l}\cup
	\left(\mathcal{R}_{i,n}\setminus\bigcup_{\left(R_{1},R_{2}\right)\in\mathcal{C}}\left\{ R_{1},R_{2}\right\} \right)\cup \mu(\mathscr{M}_i)
	\end{align}
\end{enumerate}



\subsection{Termination Condition}
The algorithm terminates at step $k$ if
\[
is_{TOT,k},is_{TOT,k-1},\dots,is_{TOT,k-\beta}\le\gamma\cdot is_{TOT,k-\left(\beta-1\right)}
\]
 where $\beta$ and $\gamma$ are selected by the experimenter. 
The algorithm terminates when in the last
$\beta$ iterations not a great ``amount of innovation'' has been
discovered. Therefore it is very probably that carrying on iterating
will not considerably change the Pareto-front approximation already formed. 

It is not recommended to chose $\beta=1$, because it is possible that the
$k-1$-th and $k$-th Pareto front approximations are not very different but next ones could be. In other words,
it is unsafe to terminate as soon as the Pareto front does not considerably change
between two consecutive eras, thus a strategy taking into
account the history of the Pareto fronts has been adopted in the
proposed approach.
%!!! IN INGLESE E' PIU' CORRETTO DIRE: ``IT IS POSSIBLE THAT THE PARETO
%FRONTS WERE NOT VERY DIFFERENT''? (CON ``WERE'' AL POSTO DI ``ARE'')


Observe that merging and splitting operation are dynamic processes. A portion
of the parameter space may be populated by a large number of small
regions in some eras, but can be covered by a small number of large
regions in next eras. This is a desirable property of PS, since, after a number
of eras of intense exploration of a portion of configuration space, this
exploration has turned to be enough thorough and so it may be time for
other portions to be minutely scanned.

A pseudo-code representation of PS algorithm is in Alg.\ref{alg:algo}

\begin{algorithm}[t]
	\small
	\caption{PS Algorithm}
	\label{alg:algo}
	  \SetKwInOut{Input}{Input}\SetKwInOut{Output}{Output}
	  \Input{S}
	  \Output{ $\tilde{\mathscr{P}}$ // Approximation of Pareto-set}
%
	  \nl //Initialization \\
	  \nl $\mathscr{R}_0\Leftarrow
				\left\{ V_{1}\times\dots\times V_{n}\right\} $; \\
	  \nl $\mathscr{P}_0\Leftarrow $ extract\_non\_dominated($test_0$);\\
	  \nl \\
	  \nl $i\Leftarrow 0$;\\
	  \nl \While{ !termination\_condition } {
			// Iteration steps\\
			\nl $test_i=\emptyset$;\\
	  		\nl \ForEach {$R \in \mathscr{R}_i$}{
				\nl $\mathscr{N}_{R,i} \Leftarrow$ configurations in $R$ not evaluated so far;\\
				\nl $|\mathscr{N}_{R,i}|\leq K/|\mathscr{R}_i|$? \\
						$\mbox{\,\,\,\,\,\,\,\,\,\,\,\,\,\,\,\,\,\,}test_{R,i}\Leftarrow \mathscr{N}_{R,i}$:\\
						\mbox{\,\,\,\,\,\,\,\,\,\,\,\,\,\,\,\,\,\,}$test_{R,i}\Leftarrow K/|\mathscr{R}_i|$ randomly selected configurations from $\mathscr{N}_{R,i}$;\\
			\nl $test_i\Leftarrow test_i \cup test_{R,i}$

			}

			\nl simulate ($test_i$);\\
			\nl $\mathscr{P}_i \Leftarrow 
				\mathscr{P}\left(test_{i}\cup\mathscr{P}_{i-1}\right)$;\\
			\nl \ForEach{$\mathbf{c}^*\in\mathscr{P}_i$} {
				\nl $is\left(\mathbf{c}^*\right) \Leftarrow \min\left\{ \left.s\left(\mathbf{c}\rightarrow\mathbf{c} \right)\right|\mathbf{c}\in\mathscr{P}_{i-1}\right\}$;\\
				}
			\nl \ForEach{$R\in\mathscr{R}_i$} {
				\nl $is\left(R\right)=\sum\left\{ \left.is\left(\mathbf{c}\right)\right|\mathbf{c}\in\mathscr{P}_{i}\cap R\right\} $;
			}

			\nl $is_{TOT,i}=\sum_{R\in\mathcal{R}_{i}}is\left(R\right)$ \\
			\nl $is_{av,i}=is_{TOT,i} / |\mathcal{R}_i|$\\
			\nl \\ //Region classification\\
			\nl $\mathcal{R}_{i,h}=\left\{ \left.R\in\mathcal{R}_{i}\right|is\left(R\right)>\alpha\cdot is_{av,i-1}\right\}$; \\
			\nl $\mathcal{R}_{i,l}=\left\{ \left.R\in\mathcal{R}_{i}\right|0<is\left(R\right)\le\alpha\cdot is_{av,i-1}\right\} $; \\
			\nl $\mathcal{R}_{i,n}=\left\{ \left.R\in\mathcal{R}_{i}\right|is\left(R\right)=0\right\} $;
			
			\nl \\ // Splitting and merging operations\\
			\nl $\psi(\mathscr{R}_{i,h}) \Leftarrow \lbrace 
					\psi(R) | R\in \mathscr{R}_{i,h}	
				\rbrace$; \\
			\nl construct $\mathscr{M}_i$; // the set of pair of contiguous no innovation regions \\
			\nl $\mu(\mathscr{M}_i) = \bigcup_{(R_1,R_2)\in\mathscr{M}_i} \mu(R_1,R_2)$;\\
			\nl\\ // Prepare the configuration space partition for the next iteration \\
			\nl $\mathcal{R}_{i+1} \triangleq
	\psi\left(\mathscr{R}_{i,h}\right) \cup \mathcal{R}_{i,l}\cup
	\left(\mathcal{R}_{i,n}\setminus\bigcup_{\left(R_{1},R_{2}\right)\in\mathcal{C}}\left\{ R_{1},R_{2}\right\} \right)\cup \mu(\mathscr{M}_i)$;\\
		}
	\nl $\tilde{\mathscr{P}} \Leftarrow \mathscr{P}_i$
\end{algorithm}
%


\comment{AA: riferirsi a \cite{zitzler_ec00} Sec. 8 per vedere l'impatto di K.}
